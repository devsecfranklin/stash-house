% !TeX document-id = {883512ec-ff7f-425c-a218-0c58e974f02f}
% !TeX root = stash.tex
% !TeX TXS-program:compile = txs:///pdflatex/[--shell-escape]
% !TeX encoding = UTF-8
% !TeX spellcheck = en_US
% https://orcid.org/0000-0003-4586-8500

% Other possible values are: 1610, 149, 54, 43 and 32.
% By default, it is to 128mm by 96mm(4:3)
\documentclass[aspectratio=169]{beamer}
%\setbeamertemplate{headline navigation symbols}{} 	% no navigation symbols
\usetheme{Warsaw}
\usecolortheme{seahorse}
\usepackage[absolute,overlay]{textpos} % Text positioning


%Information to be included in the title page:
\title{Stash House}
\subtitle{Collect and Hide Local Secrets}
\author{Franklin D.}
\institute{pale shadow}
\date{2023}

%%%%%%%%%%%%%%%%%%%%%%%%%%%%%%%%%%%%%%%%% SECTION

\usepackage[T1]{fontenc}
\usepackage{lmodern}
\usepackage{tikz}
\usetikzlibrary{positioning,calc,shadings}

\tikzset{section number/.style={
		inner sep=0pt,
		draw=none,
	},
	section/.style={
		inner sep=0pt,
		draw=none,
		%text width=\the\dimexpr\paperwidth-3.8em\relax,
		text=blue,
		align=center
	},
	subsection/.style={
		inner sep=0pt,
		draw=none,
		%text width=\the\dimexpr\paperwidth-3.8em\relax,
		text=gray,
		align=center
	}
}

\makeatletter
\def\sectionsubtitle#1{\gdef\@sectionsubtitle{#1}}
\AtBeginSection[]{%
	\begingroup
	% Lighter, beamer native version
	\setbeamercolor{background canvas}{bg=gray!30}
	%
	\begin{frame}
		\begin{tikzpicture}[remember picture,overlay]
			\coordinate[yshift=-10mm] (rectsouthwest) at (current page.north west);
			\coordinate[yshift=-10mm] (rectsoutheast) at (current page.north east);
			\fill[white] (current page.north west) rectangle (rectsoutheast);
			\shade [left color=gray,right color=white] (rectsouthwest) rectangle +(\paperwidth,-0.02);
			\node[section,anchor=center] at (current page.center) (title) {\fontsize{20}{20}\selectfont\insertsectionhead};
			\node[subsection,below=5mm of title]  (subtitle) {\@sectionsubtitle};
		\end{tikzpicture}
	\end{frame}
	\gdef\@sectionsubtitle{}
	\endgroup
}
\makeatother

%%%%%%%%%%%%%%%%%%%%%%%%%%%%%%%%%%%%%%%%%  Notes pages

% These slides also contain speaker notes. You can print just the slides,
% just the notes, or both, depending on the setting below. Comment out the want
% you want.

%\setbeameroption{hide notes} % Only slides
%\setbeameroption{show only notes} % Only notes
%\setbeameroption{show notes on second screen=right} % Both

% To give a presentation with the Skim reader (http://skim-app.sourceforge.net) on OSX so
% that you see the notes on your laptop and the slides on the projector, do the following:
%
% 1. Generate just the presentation (hide notes) and save to slides.pdf
% 2. Generate onlt the notes (show only nodes) and save to notes.pdf
% 3. With Skim open both slides.pdf and notes.pdf
% 4. Click on slides.pdf to bring it to front.
% 5. In Skim, under "View -> Presentation Option -> Synhcronized Noted Document"
%    select notes.pdf.
% 6. Now as you move around in slides.pdf the notes.pdf file will follow you.
% 7. Arrange windows so that notes.pdf is in full screen mode on your laptop
%    and slides.pdf is in presentation mode on the projector.

% Give a slight yellow tint to the notes page
%\setbeamertemplate{note page}{\pagecolor{yellow!5}\insertnote}\usepackage{palatino}
%\setbeamertemplate{note page}{\pagecolor{yellow!5}\vfill\insertnote\vfill}

%\setbeamertemplate{note page}{%
%	\pagecolor{yellow!5}
%	\vfill
%	\begin{minipage}[c][\textheight][t]{\textwidth}
%		{\usebeamerfont{frametitle}\usebeamercolor[fg]{frametitle}\insertframetitle\par}
%		\insertnote
%	\end{minipage}
%}

% customize the caption on figures
\setbeamertemplate{caption}[numbered]
\setbeamerfont{caption}{size=\large}
\setbeamercolor{caption}{fg=black}
\setbeamercolor{caption name}{fg=black}

\begin{document}

{
\usebackgroundtemplate{\includegraphics[width=\paperwidth]{../static/images/Gemini\_Generated\_Image\_57v54w57v54w57v5.png}}%
\begin{frame}
	\titlepage
    \note[item]{https://www.youtube.com/watch?v=DYDT165LGBY}
\end{frame}
}

%\begin{frame}
%	\frametitle{Table of Contents}
%	\tableofcontents
%\end{frame}

% you can uncomment one of these for the whole doc, or add at the start of each section as desired
\usebackgroundtemplate{\includegraphics[width=\paperwidth]{../static/images/field.jpg}}
%\usebackgroundtemplate{\includegraphics[width=\paperwidth]{../images/landscape.jpg}}
%\usebackgroundtemplate{\includegraphics[width=\paperwidth]{../images/tree.jpg}}


\sectionsubtitle{The Lost Art of Keeping a Secret}
\section{Introduction}

\begin{frame}
	\frametitle{Step One: Admit That You Have a Problem}

\begin{columns}
	\column{0.5\textwidth}

    Most of us have a (reasonable?) expectation of trust for the files on our local machine. And so, we leave things like saved passwords and other credentials and secrets pasted into text files for quick access.
	\column{0.5\textwidth}

	\begin{figure}
		\centering
		\includegraphics[width=0.77\textwidth]{../static/images/problem.jpg}
		\caption{You}
		\label{fig:question}
	\end{figure}

	\end{columns}

	\note[item]{Yes you do this}
	\note[item]{No it's not good}

\end{frame}

\begin{frame}
	\frametitle{Project Goals}
    \begin{itemize}
    	\item Locate unprotected credentials on the local host
        \item Encrypt and safely store these credentials and other secrets.
        \item Access to credential store from multiple locations.
    \end{itemize}

\note[item]{Clean up the ones we know about}
\note[item]{Of course there will always be gaps and edge cases, just try to improve over time}
\note[item]{What prior art is out there}

\end{frame}

\sectionsubtitle{What is it you want to protect?}
\section{Finding Those Creds}

\begin{frame}
	\frametitle{What Do We Mean by Secrets, Exactly}
	\begin{itemize}
		\item Credentials for example a username/password for that lab host that you only need for a couple weeks.
		\item Text files that are used as keys.
	\end{itemize}

    \note[item]In my case I had a whole folder of text files that I'd collected of some time.
\end{frame}

\begin{frame}
	\frametitle{Finding Tokens}
	\begin{itemize}
		\item Look around your local machines for tokens and credentials.
		\item Use some automation to help you find them.
	\end{itemize}
	\note[item]{There are some known places we can look}
	\note[item]{Clean up the ones we know about}
\end{frame}

\begin{frame}
	\frametitle{Demo Script}
	\begin{itemize}
		\item Here is a small tool that you can use/modify for your system
	\end{itemize}
	\note[item]{Do a small demo here}
\end{frame}

\sectionsubtitle{We found the secrets, now what?}
\section{Hide Your Goodies}

\begin{frame}
	\frametitle{Saving Simple Tokens}
	\begin{itemize}
		\item Encrypt the tokens and push them into RCS.
	\end{itemize}
\end{frame}

\begin{frame}
	\frametitle{Setting Up the Stash House}
	\begin{itemize}
		\item Create a repository on GitHub or other revision control system.
		\item Install the framework with shell script.
		\item Set up your GPG key.
		\item Add items to DB, remove plain text tokens and secrets.
	\end{itemize}
\end{frame}

\begin{frame}
	\frametitle{Saving Simple Tokens}
	\begin{itemize}
		\item Encrypt the tokens and push them into RCS.
	\end{itemize}
    \note[item]The first kind of secret we want to save is a "simple" password, basically a string.
\end{frame}

\begin{frame}
	\frametitle{Saving Multi-line Tokens}
	\begin{itemize}
		\item Encrypt the tokens and push them into RCS.
	\end{itemize}
    \note[item]We can also save multi-line secrets, GCloud JSON for example.
\end{frame}

\begin{frame}
	\frametitle{Backing up Tokens to RCS}
	\begin{itemize}
		\item Encrypt the tokens and secrets
		\item Push everything into revision control.
	\end{itemize}
\end{frame}

\sectionsubtitle{How to Use What You Built}
\section{Using Your Stashed Tokens}

\begin{frame}
	\frametitle{Using Tokens}
	\begin{itemize}
		\item Now you can use the secrets in your project without exposing them.
	\end{itemize}
\end{frame}

\begin{frame}
	\frametitle{Considerations}
	\begin{itemize}
		\item You need your GPG key on the local machine to encrypt, decrypt, and use the secrets.
	\end{itemize}
\note[item]{Don't be lazy, you still have to use a password manager}
\end{frame}

\end{document}