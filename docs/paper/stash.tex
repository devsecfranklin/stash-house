% !TeX encoding = UTF-8
% !TeX spellcheck = en_US
% !TeX root = stash.tex
% !TeX TXS-program:compile = txs:///pdflatex/[--shell-escape]
% https://orcid.org/0000-0003-4586-8500


\documentclass[11pt]{report}
%\documentclass[11pt, twocolumn]{report}

\usepackage{lipsum} %Package to generate latin text (Only for testing)

\usepackage{amsmath} %AMS packages for mathematical symbols
\usepackage{amsfonts}
\usepackage{amssymb}

\usepackage{graphicx} %package to include pictures and images
\usepackage[margin=1in, includefoot, includehead]{geometry} %Package to setup page layout

\usepackage{minted} % code blocks
% \usepackage[hidelinks]{hyperref} %Uses hyperlinks in PDF (optional)
\usepackage{hyperref}

%Header and Footer Stuff
\usepackage{fancyhdr}
\pagestyle{fancy}
\fancyhead{}
\fancyfoot{}
\fancyhead[R]{\myTitle}
%\fancyfoot[L]{Credential Management Tooling}
%\fancyfoot[R]{\thepage}
\fancyfoot[R]{\includegraphics[scale=0.50]{../../static/images/ucd-wide.jpg}}
\renewcommand{\footrulewidth}{1pt}

% Document Variables
\newcommand{\myTitle}{Stash House}
\newcommand{\myName}{Franklin Diaz}
\newcommand{\myOrg}{University of Colorado}
\newcommand{\myDate}{September 25th, 2025}

\begin{document}

% frontmatter: half title, title page, colophon (copyright page), epigraph, toc, preface, acknowledgements
%\frontmatter{}
	\begin{titlepage}
		\begin{center}

			\line(1,0){400}\\
			[2mm]
			\textsc{\Large{\myTitle}} \\
			\line(1,0){400} \\
			[3in]
		\end{center}
		\begin{center}
			\textsc{\Large \myName}	\\
			%\myOrg\\
			[2.5in]
			\myDate\\

		\includegraphics[scale=0.75]{../../static/images/ucd-wide.jpg}\\
			% \huge{\textbf{\myOrg}} \\
			%[0.25in]
		\end{center}
	\end{titlepage}


%If need adding sections, please uncomment next lines
%Before TOC
\pagenumbering{roman}
\section*{Summary}
\addcontentsline{toc}{section}{\numberline{}Summary}
In 2023 started to focus on issues arising from a "dirty" local development
environment, and precisely what to do about it. This paper is an evaluation of using freely available
Open Source tools and a few easy to remember patterns we can work in a much more efficient and secure manner.

\begin{itemize}
   	\item First we look at how to set up a workable solution for storing credentials.
	\item Next we do some local search and cleanup to get our systems in order.
	\item Finally, we will see how we can still easily use these credentials though
		they remain encrypted at rest and in transit to our "vault" from now on.
\end{itemize}

\cleardoublepage


%If need table of contents, please uncomment next lines
%Table of contents
\tableofcontents
\thispagestyle{empty}
\cleardoublepage
\setcounter{page}{1}

%Start your text here
\pagenumbering{arabic}

\chapter{Introduction}

While working as a developer, the need frequently arises to work in a ``sandbox'' type environment, where security
controls are often skipped in favor of speed and efficiency. During the course of the software development life cycle,
the need arises to give access to sensitive credentials including databases, cloud accounts, certificates, and many more.

\chapter{Safety Considerations for Data at Rest}

The main premise of this paper is to improve the hygiene of local credential storage. With this goal in mind,
we approach the question of how to meet our storage goal.

\section{Local Back End Storage}

While the tool chain set forth in this paper will use the \href{https://gnupg.org/}{GNU Privacy Guard} to encrypt
user secrets for safe storage, an additional security safeguard can be implemented by simply storing your data in a ``local'' fashion. That is to say, the encrypted data stays within the Enterprise Network or Administrative Domain, with the goal of reducing it's accessibility those who are unauthorized.



\section{Distributed Back End Storage}



% include foks paper in the bibliography

\end{document}
