% !TeX encoding = UTF-8
% !TeX spellcheck = en_US
% !TeX root = stash.tex
% !TeX TXS-program:compile = txs:///pdflatex/[--shell-escape]
% https://orcid.org/0000-0003-4586-8500


\documentclass[11pt]{report}
%\documentclass[11pt, twocolumn]{report}

\usepackage{lipsum} %Package to generate latin text (Only for testing)

\usepackage{amsmath} %AMS packages for mathematical symbols
\usepackage{amsfonts}
\usepackage{amssymb}

\usepackage{graphicx} %package to include pictures and images
\usepackage[margin=1in, includefoot, includehead]{geometry} %Package to setup page layout

\usepackage{minted} % code blocks
% \usepackage[hidelinks]{hyperref} %Uses hyperlinks in PDF (optional)
\usepackage{hyperref}

%Header and Footer Stuff
\usepackage{fancyhdr}
\pagestyle{fancy}
\fancyhead{}
\fancyfoot{}
\fancyhead[R]{\myTitle}
\fancyfoot[L]{Credential Management Tooling}
\fancyfoot[R]{\thepage}
\renewcommand{\footrulewidth}{1pt}

% Document Variables
\newcommand{\myTitle}{Stash House}
\newcommand{\myName}{franklin d.}
\newcommand{\myOrg}{pale shadow}
\newcommand{\myDate}{September 25th, 2025}

\begin{document}

	\begin{titlepage}
		\begin{center}
			\includegraphics[scale=0.20]{../static/images/new_logo.png}\\
			% \huge{\textbf{\myOrg}} \\
			[0.25in]

			\textbf{\Large{\myOrg}}\\
			\Large{recreational computing enthusiasts}\\
			[1in]

			\line(1,0){400}\\
			[2mm]
			\textsc{\Large{\myTitle}} \\
			\line(1,0){400} \\
			[1in]
		\end{center}
		\begin{center}
			\textsc{\Large \myName}	\\
			\myOrg\\
			[1in]
			\myDate
		\end{center}
	\end{titlepage}

	%If need adding sections, please uncomment next lines
	%Before TOC
	\pagenumbering{roman}
	\section*{Summary}
	\addcontentsline{toc}{section}{\numberline{}Summary}
    In this project paper we will consider the issue of a "dirty" local development
    environment, and precisely what to do about it. Using freely available
    Open Source tools and a few easy to remember patterns we can work in a
    much more efficient and secure manner.

    \begin{itemize}
    	\item First we look at how to set up a workable solution for storing credentials.
		\item Next we do some local search and cleanup to get our systems in order.
		\item Finally, we will see how we can still easily use these credentials though
			they remain encrypted at rest and in transit to our "vault" from now on.
    \end{itemize}

	\cleardoublepage


	%If need table of contents, please uncomment next lines
	%Table of contents
	\tableofcontents
	\thispagestyle{empty}
	\cleardoublepage
	\setcounter{page}{1}

	%Start your text here
	\pagenumbering{arabic}

	\chapter{Introduction}

    \lipsum[1]

	\section{Setting Up}

	The same repo where this paper can be downloaded has the tools described in this and future chapters.

\chapter{Finding Secrets on Local Machine}

There is a BASH shells script that can be used to scan the local files for
unsecured credentials that may be a candidate for encryption and safe storage.

\begin{minted}{cpp}
    #include <iostream>

    int main() {
        std::cout << "Hello, C++!" << std::endl;
        return 0;
    }
\end{minted}

\chapter{Storing Secrets}

	\lipsum[1]

	\chapter{Using Secrets}

	\lipsum[1]

\end{document}
