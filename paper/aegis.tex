% Template for PhD reports
% By: Roberto Contreras
% This document has been cleaned up for better readability and structure.

\documentclass[11pt]{article}

% --- LaTeX Packages ---
\usepackage{lipsum} % For generating placeholder text
\usepackage{amsmath, amsfonts, amssymb} % AMS packages for mathematical symbols
\usepackage{graphicx} % To include pictures and images
\usepackage[margin=1in, includefoot, includehead]{geometry} % To set up page layout
\usepackage[hidelinks]{hyperref} % Uses hyperlinks in PDF (optional)
\usepackage{fancyhdr} % For custom headers and footers

% --- Document Variables ---
\newcommand{\myTitle}{Stash}
\newcommand{\myName}{Enter Name}
\newcommand{\myClass}{Enter Name}
\newcommand{\myTecher}{Enter Name}
\newcommand{\myDate}{December 28, 2022}

% --- Header and Footer Setup ---
\pagestyle{fancy}
\fancyhead{} % Clear header
\fancyfoot{} % Clear footer
\fancyhead[R]{\myTitle}
\fancyfoot[L]{Stash: Credential Management Tooling}
\fancyfoot[R]{\thepage}
\renewcommand{\footrulewidth}{1pt}

% --- Document Start ---
\begin{document}
	
	% --- Title Page ---
	\begin{titlepage}
		\begin{center}
			\includegraphics[scale=0.75]{../images/new_logo.jpeg}\\
			\huge{\textbf{DE:AD:10:C5}} \\
			[0.25in]
			
			\textbf{\Large{Instituto de Ingeniería y Tecnología}}\\
			\Large{Departamento de Ingeniería Industrial y Manufactura}\\
			\Large{Doctorado en Tecnología}\\
			[1in]
			
			\line(1,0){400}\\
			[2mm]
			\textsc{\Large{\myTitle}} \\
			\line(1,0){400} \\
			[1in]
		\end{center}
		\begin{center}
			\textsc{\Large \myName} \\
			\myClass\\
			\myTecher\\
			[1in]
			\myDate
		\end{center}
	\end{titlepage}
	
	% --- Summary Section ---
	\pagenumbering{roman}
	\section*{Summary}
	\addcontentsline{toc}{section}{\numberline{}Summary}
	
	[cite_start]In this project paper, we will consider the issue of a "dirty" local development environment, and precisely what to do about it[cite: 40]. [cite_start]Using freely available Open Source tools and a few easy-to-remember patterns, we can work in a much more efficient and secure manner[cite: 41].
	
	\begin{itemize}
		[cite_start]\item First, we look at how to set up a workable solution for storing credentials[cite: 42].
		[cite_start]\item Next, we do some local search and cleanup to get our systems in order[cite: 43].
		[cite_start]\item Finally, we will see how we can still easily use these credentials though they remain encrypted at rest and in transit to our "vault" from now on[cite: 44].
	\end{itemize}
	
	\cleardoublepage
	
	% --- Table of Contents ---
	\tableofcontents
	\thispagestyle{empty}
	\cleardoublepage
	\setcounter{page}{1}
	\pagenumbering{arabic}
	
	% --- Main Body Sections ---
	\section{Introduction}
	\lipsum[1]
	\section{Setting Up}
	\section{Finding Secrets on Local Machine}
	\section{Storing Secrets}
	\section{Using Secrets}
	
\end{document}